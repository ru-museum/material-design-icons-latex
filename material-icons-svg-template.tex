\documentclass[10pt]{ltjarticle}
\usepackage{fontspec}

% LINK
\usepackage{url}
\usepackage{hyperref}

% 色の使用
\usepackage{xcolor}
\definecolor{code}{HTML}{800000}
\definecolor{mylinkcolor}{RGB}{3, 112, 145} %{65, 145, 3} % 色定義
\hypersetup{
    colorlinks=true,
    urlcolor=mylinkcolor, % 定義された色
}

% 表の使用
\usepackage{float}

% MATERIAL-ICONS
\usepackage{materiaicons} 
\usepackage{svg} 

% FONT-SIZE 定義
\def\fs#1{\fontsize{#1pt}{14pt}\selectfont}
% {\fs{24.88} 

\title{{\fs{24.88}Material Design Icons\\ Template for \LaTeX{}}\\ \vspace{10mm}{\fs{12} \textbf{[ SVG 版 ]}}}
\author{}
\date{\today}

% PNG サンプル定義
\newcommand{\mdBuildPng}{\includegraphics[width=1em]{./material-icons/build.png}}

\begin{document}

\maketitle

\section{material-design-icons (Google)}

\subsection{表示例}

\begin{table}[H]
\centering
% \caption{アイコン表示例}
\begin{tabular}{ll}
{\fs{20} \mdBuild} \hspace{0.6em} mdBuild & {\fs{20} \mdAddComment} \hspace{0.6em} mdAddComment\\
{\fs{20} \mdAccessibility} \hspace{0.6em} mdAccessibility & {\fs{20} \mdPDIcon} \hspace{0.6em} mdPDIcon\\
\end{tabular}
\end{table}

\subsection{SVG と PNG との表示比較}

\begin{itemize}
  \item PNG は拡大すると輪郭が不鮮明に表示されます。
\end{itemize}

\begin{table}[H]
\centering
% \caption{SVG と PNG との表示比較}
\begin{tabular}{lll}
フォントサイズ & SVG & PNG\\ 
\hline
\vspace{-2mm} & \\
{\fs{20}20pt} & {\fs{20} \mdBuild} & {\fs{20} \mdBuildPng}\\
{\fs{40}40pt} & {\fs{40} \mdBuild} & {\fs{40} \mdBuildPng}\\
{\fs{60}60pt} & {\fs{60} \mdBuild} & {\fs{60} \mdBuildPng}\\
\end{tabular}
\end{table}


\newpage

\section{使用方法}

\begin{itemize}
  \item 一括ダウンロードで利用する場合は SVG ファイル(src 内)は全て同名(24px.svg)となっている為、ファイル名を変更する必要があります。
\end{itemize}

\subsection{ICON ファイル(.svg)のダウンロード。}
\href{https://fonts.google.com/icons}{Material Design Icons}からアイコンをクリックし SVG ファイルを所定のフォルダに保存します。

\subsection{ファイル名はを簡便なものに変更して置きます。}
\hspace{8mm}例: thumb\_up\_black\_24dp.svg → thumb\_up.svg 

\subsection{パッケージファイルの編集。}
\textbf{materialicons.sty}\\  
{\fs{8}
\hspace{4mm}\% フォントサイズ:変更可能\vspace{1mm}\\
\hspace{4mm}\textbackslash def\textbackslash @icon\{\textbackslash includesvg[width=\textbf{1em}]\}\vspace{2mm}\\
\hspace{4mm}\% アイコンファイル(.svg)フォルダ:変更可能\vspace{1mm}\\
\hspace{4mm}\textbackslash def\textbackslash @dir\{\textbf{./material-icons}\}\vspace{2mm}\\
\hspace{4mm}\% 以下にファイル名に従い追加します\vspace{1mm}\\
\hspace{4mm}\textbackslash newcommand\{\textcolor{code}{\textbf{\textbackslash mdBuild}}\}\{\textbackslash@icon\{\textbackslash@dir/\textcolor{code}{\textbf{build}}.svg\}\}\\
\hspace{4mm}\textbackslash newcommand\{\textcolor{code}{\textbf{\textbackslash mdAccessibility}}\}\{\textbackslash@icon\{\textbackslash@dir/\textcolor{code}{\textbf{accessibility}}.svg\}\}
}

\subsection{TEX 内での表示。}
\% パッケージの読込\\
\hspace{4mm}\textbackslash usepackage\{materiaicons\}\\
\hspace{4mm}\textbackslash usepackage\{svg\} \% 必須\\

\% フォントサイズのマクロを定義()\\
\hspace{4mm}\textbackslash def\textbackslash fs\#1\{\textbackslash fontsize\{\#1pt\}\{14pt\}\textbackslash selectfont\}
  
\begin{table}[H]
\centering
% \caption{アイコンの表示}
\begin{tabular}{ll}
デフォルト & サイズ指定\\
\hline
\textbackslash mdBuild & \{\textbackslash fs\{20\} \textbackslash mdBuild\}\\
\mdBuild & {\fs{20} \mdBuild}\\
\end{tabular}
\end{table}
 

\end{document}
